\documentclass{article}
\usepackage[utf8]{inputenc}
\usepackage[acronym, toc]{glossaries}
\usepackage{lmodern}
\usepackage[rgb]{xcolor}
\usepackage[author={Michał Małaj}]{pdfcomment}

\usepackage[backend=bibtex,
bibencoding=utf8
%style=alphabetic
%style=reading
]{biblatex}
\addbibresource{przypisy/przypisy.bib}

\usepackage{hyperref}
\hypersetup{
    colorlinks=true,
    linkcolor=blue,
    filecolor=magenta,      
    urlcolor=cyan,
}
 
\urlstyle{same}

\makeglossaries

\newglossaryentry{latex}
{
    name=latex,
    description={Is a mark up language specially suited for scientific documents}
}

\newglossaryentry{energii promieniowania}
{
    name=energia promieniowania,
    description={Natężenie promieniowania słonecznego docierającego do górnych granic atmosfery określone jest przez stałą słoneczną. Wielkość ta jest zdefiniowana dla średniej odległości Ziemia-Słońce i wynosi około 1366,1 W/m2. }
}

\newglossaryentry{}
{
    name=latex,
    description={Is a mark up language specially suited for scientific documents}
}


\newglossaryentry{maths}
{
    name=mathematics,
    description={Mathematics is what mathematicians do}
}

\newglossaryentry{konwersja}
{
    name=konwersji fotowoltaicznej,
    description={Fotowoltaiczna przemiana energii promieniowania słonecznego zachodzi wówczas, gdy energia promieniowania słonecznego jest zamieniana w sposób czysto elektronowy na energię elektryczną. Efekt fotowoltaiczny polega na powstawaniu siły elektromotorycznej w materiale półprzewodnikowym, złączu p-n, w wyniku oświetlania go promieniowaniem o odpowiedniej długości fali. W ogniwach fotowoltaicznych jest wykorzystywane zjawisko fotoelektryczne wewnętrzne zaporowe zachodzące w półprzewodnikach np. w krzemie.}
}

\newacronym{gcd}{GCD}{Greatest Common Divisor}

\newacronym{lcm}{LCM}{Least Common Multiple}

\begin{document}

\tableofcontents

\section{Test na egzamin z fotowoltaiki}

Egzamin fotowoltaika zagadnienia test

Średnia ilość \pdfmarkupcomment[markup=Squiggly,color=green]{energii promieniowania}{Natężenie promieniowania słonecznego docierającego do górnych granic atmosfery określone jest przez stałą słoneczną. Wielkość ta jest zdefiniowana dla średniej odległości Ziemia-Słońce i wynosi około 1366,1 W/m2.}.
\footfullcite{foo12} dla województwa lubelskiego wynosi: \textbf{1000 Wm2}

Panele przetwarzają energię: \textbf{za pomocą \pdfmarkupcomment[markup=Squiggly,color=green]{konwersji fotowoltaicznej}{Fotowoltaiczna przemiana energii promieniowania słonecznego zachodzi wówczas, gdy energia promieniowania słonecznego jest zamieniana w sposób czysto elektronowy na energię elektryczną. Efekt fotowoltaiczny polega na powstawaniu siły elektromotorycznej w materiale półprzewodnikowym, złączu p-n, w wyniku oświetlania go promieniowaniem o odpowiedniej długości fali. W ogniwach fotowoltaicznych jest wykorzystywane zjawisko fotoelektryczne wewnętrzne zaporowe zachodzące w półprzewodnikach np. w krzemie.}}\footfullcite{konwersja}

Instalacja autonomiczna to instalacja \textbf{\pdfmarkupcomment[markup=Squiggly,color=green]{off grid}{}} czyli wyłączona z sieci \footfullcite{offgrid}

Najwyższą sprawność ogniw fotowoltaicznych mają: \textbf{\pdfmarkupcomment[markup=Squiggly,color=green]{ogniwa hybrydowe}{}}

Jakiego typu diody zabezpieczają panele fotowoltaiczne połączone szeregowo przez wpływem prądu zwrotnego: \textbf{\pdfmarkupcomment[markup=Squiggly,color=green]{diody bocznikujące}{}}

Ogniwa fotowoltaiczne umieszczone na kątach stałych powinny być pochylone  pod kątem: \textbf{\pdfmarkupcomment[markup=Squiggly,color=green]{35 stopni}{}}

Maksymalna sprawność modułu fotowoltaicznego wynosi: \textbf{\pdfmarkupcomment[markup=Squiggly,color=green]{15-18 \%}{}}

Uzyskiwana \textbf{\pdfmarkupcomment[markup=Squiggly,color=green]{moc szczytowa}{}} jest największa gdy:\textbf{płaszczyzna generatora jest skierowana do promieni słonecznych prostopadle}

Wyjście paneli do inwertera jest:  \textbf{\pdfmarkupcomment[markup=Squiggly,color=green]{stałoprądowe DC}{}} 

Najmniejszy element fotowoltaiczny to: \textbf{ogniwo}

Za pomocą jakich urządzeń wykonuje się pomiar energii elektrycznej: \textbf{licznik energii elektrycznej, dwukierunkowy}

Czy budowa instalacji fotowoltaicznej wymaga pozwolenia na budowę:  \textbf{na zgłoszenie}

Pierwiastek wykorzystywany do budowy ogniw: \textbf{krzem}

Jakie są zabezpieczenia w instalacjach fotowoltaicznych: \textbf{ograniczniki przepięć, instalacja odgromowa, instalacja przeciążeniowa}

Do czego służy inwerter: \textbf{konwertuje prąd stały DC na prąd zmienny AC}

Wyposażenie instalacji  autonomicznej wyspowej: \textbf{panel fotowoltaiczny, regulator ładowania, akumulator, inwerter dla systemu off grid}

Sprawność energetyczna fotowoltaiczna ogniwa: \textbf{zwiększa się pod wpływem wzrostu natężenia oświetlenia}

String jest to: \textbf{zespół połączonych szeregowo baterii - paneli}

Czy panele fotowoltaiczne mogą wytwarzać 230 V? \textbf{nie mogą wytwarzać 230 V}

Przetwornicę do zmiany napięcia stałego na zmienny określa się jakim symbolem: \textbf{DC/AC}

Odległość paneli od krawędzi dachu: \textbf{od 0,6 do 1 m}

Nasłonecznienie to wielkość fizyczna: \textbf{określająca średnią moc promieniowania przypadającą na jednostkę powierzchni}

Polska znajduje się w strefie nasłonecznienia: \textbf{do 900 do 1000 kWh/m2}

Co to jest mikroinstalacja: \textbf{instalacja OZE o łącznej mocy energii elektrycznej przyłączona do sieci energii elektrycznej o napięciu znamionowym nie większym niż 110 kW}

Rodzaje systemów fotowoltaicznych: \textbf{on grid, off grid}

Moc modułów fotowoltaicznych podajemy w: \textbf{Wp - watt peak}

Wzrost temperatury paneli fotowoltaicznych jest wtedy kiedy:
Maleje moc wytwarzania energii przez panel fotowoltaiczny

Jaka jest różnica między falownikiem a inwerterem:
Nie ma różnicy, to to samo, falownik polska nazwa inwertera

Co oznacza skrót BIPV ang. Buidling-Integrated Photovoltaics
Instalacja fotowoltaiczna zintegrowana z budynkiem

Jakie rodzaje połączeń stosowane są wewnątrz paneli fotowoltaicznych:
mieszane
Do produkcji ogniw fotowoltaicznych wykorzystujemy:
krzem, arsenek galu, tellurek kadmu, (wszystkie odpowiedzi są dobre)

Najwyższą sprawność energetyczną osiągają ogniwa:
Monokrystaliczne

Sprawność energetyczna fotoogniwa :
Zwiększa się wraz ze wzrostem natężenia oświetlenia

W instalacjach fotowoltaicznych wykorzystuje się obwody:
AC/DC

Instalacja przyłączona do sieci:
On grid

Wyjście inwertera jest:

Zmiennoprądowe AC

Definicja prądu DC przemiennego:
Prąd płynący w jednym kierunku o stałym poziomie natężenia, którego zmiany wartości natężenia wynoszą 0 albo są tak niewielkie , że mogą zostać zaniedbane

Maksymalna sprawność modułów fotowoltaicznych to:
15-18%

Czy można ładować akumulator w instalacji elektrycznej bez regulatora ładowania:

nie ponieważ panele mają wyższe napięcie niż akumulator

Definicja falownika:
Falownik zmienia napięcie stałe na napięcie zmienne

String:
Zespół połączonych szeregowo modułów fotowoltaicznych

Przewody leżące pod modułami PV powinny mieć wytrzymałość:

 minimum 80 stopni C

Zaznacz poprawną odpowiedź:
Dach powinien mieć od 22 do 45 stopni nachylenia

Czy istnieje konieczność modyfikacji instalacji elektrycznej do modułu do domu:
Nie ma potrzeby modyfikacji

Dopuszczalne spadki napięć na przewodach między modułami PV a falownikiem:
Muszą mieć wartość poniżej 1%

Jakie zagadnienie trzeba uwzględnić przy konstrukcji wsporczej w instalacji fotowoltaicznej 
Dodatkowe obciążenie konstrukcji dachu lub elewacji

Protokół instalacji zdawczo-odbiorczy:
Sporządzony po próbnym uruchomieniu instalacji 

Prawidłowo wykonana instalacja powinna mieć zabezpieczenie :
przepięciowe
odgromowe





























\clearpage

\section{Second Section}

\vspace{5mm}

Given a set of numbers, there are elementary methods to compute its \acrlong{gcd}, which is abbreviated \acrshort{gcd}. This process is similar to that used for the \acrfull{lcm}.


\clearpage

\printglossary

\clearpage

\printglossary[type=\acronymtype]

\end{document}
