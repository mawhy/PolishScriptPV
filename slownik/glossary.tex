\newglossaryentry{latex}
{
    name=latex,
    description={Is a mark up language specially suited for scientific documents}
}

\newglossaryentry{energii promieniowania}
{
    name=energia promieniowania,
    description={Natężenie promieniowania słonecznego docierającego do górnych granic atmosfery określone jest przez stałą słoneczną. Wielkość ta jest zdefiniowana dla średniej odległości Ziemia-Słońce i wynosi około 1366,1 W/m2. }
}

\newglossaryentry{}
{
    name=latex,
    description={Is a mark up language specially suited for scientific documents}
}


\newglossaryentry{maths}
{
    name=mathematics,
    description={Mathematics is what mathematicians do}
}

\newglossaryentry{konwersja}
{
    name=konwersji fotowoltaicznej,
    description={Fotowoltaiczna przemiana energii promieniowania słonecznego zachodzi wówczas, gdy energia promieniowania słonecznego jest zamieniana w sposób czysto elektronowy na energię elektryczną. Efekt fotowoltaiczny polega na powstawaniu siły elektromotorycznej w materiale półprzewodnikowym, złączu p-n, w wyniku oświetlania go promieniowaniem o odpowiedniej długości fali. W ogniwach fotowoltaicznych jest wykorzystywane zjawisko fotoelektryczne wewnętrzne zaporowe zachodzące w półprzewodnikach np. w krzemie.}
}

\newacronym{gcd}{GCD}{Greatest Common Divisor}

\newacronym{lcm}{LCM}{Least Common Multiple}